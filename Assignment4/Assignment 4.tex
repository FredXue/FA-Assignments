%to produce a pdf, you type: pdflatex thissample
\input{../latex-sample/00PRELIMS}  %loads 00PRELIMS.tex, which also loads zrogram.tex (all in same directory)


\title{FA HW 4}
\date{2017-10-07}
\author{Xiaoyu Xue}

\renewcommand{\thesubsection}{\thesection.\alph{subsection}}
\renewcommand{\thesection}{\arabic{section}}

\begin{document}
\maketitle

\section{Ex 2.12}
\subsection{}
\Prog\qq
\Global int $sum$ is all pre-visited vertices' dat;\qq
\Procedure{$DFS$}{v}; \p
$sum \leftarrow sum + v.dat$;\p
$v.ans \leftarrow sum$;\p
\Foreach vertex $w$ in $child[v]$ \Do\p
$DFS(w);$\p
\Endfor\qq
\Fini 

Initial Procedure Call:
\Prog\qq
\Procedure{$Driver$}{};\p
$sum \leftarrow 0$;\p
$DFS(T.root);$\qq
\Fini

\subsection{}
\Prog\qq
\Global int $sum$ is all pre-visited vertices' dat;\qq
\Procedure{$DFS$}{v}; \p
\Foreach vertex $w$ in $child[v]$ \Do\p
$DFS(w);$\p
\Endfor\p
$sum \leftarrow sum + v.dat$;\p
$v.ans \leftarrow sum$;\qq
\Fini 

Initial Procedure Call:
\Prog\qq
\Procedure{$Driver$}{};\p
$sum \leftarrow 0;$\p
$DFS(T.root);$\qq
\Fini

\subsection{}
\Prog\qq
\Global int $sum$ is all pre-visited vertices' dat;\qq
\Procedure{$DFS$}{v}; \p
$DFS(v.left);$\p
$sum \leftarrow sum + v.dat$;\p
$v.ans \leftarrow sum$;\p
$DFS(v.right);$\qq
\Fini 

Initial Procedure Call:
\Prog\qq
\Procedure{$Driver$}{};\p
$sum \leftarrow 0;$\p
$DFS(T.root);$\qq
\Fini


\subsection{}
\Prog\qq
\Global int sum is all have not yet been pre-visited vertices' dat;\qq
\Procedure{$DFS$}{v}; \p
$sum \leftarrow sum - v.dat$;\p
\Foreach vertex $w$ in $child[v]$ \Do\p
$DFS(w);$\p
\Endfor\p
$v.ans \leftarrow sum$;\qq
\Fini 

\Prog\qq
\Procedure{$initDFS$}{v}; \p
$sum \leftarrow sum + v.dat$;\p
\Foreach vertex $w$ in $child[v]$ \Do\p
$DFS(w);$\p
\Endfor\qq
\Fini 

Initial Procedure Call:
\Prog\qq
\Procedure{$Driver$}{};\p
$sum \leftarrow 0;$\p
$initDFS(T.root);$\qq
$DFS(T.root);$\qq
\Fini




\subsection{}
\Prog\qq
\Global int $sum$ is all pre-visited vertices' dat;\qq
\Procedure{$DFS$}{v}; \p
$sum \leftarrow sum + v.dat$;\p
\Foreach vertex $w$ in $child[v]$ \Do\p
$DFS(w);$\p
\Endfor\p
$sum \leftarrow sum - v.dat$;\p
$v.ans \leftarrow sum$;\qq
\Fini 

Initial Procedure Call:
\Prog\qq
\Procedure{$Driver$}{};\p
$sum \leftarrow 0;$\p
$DFS(T.root);$\qq
\Fini




\subsection{}
Those vertices that have been preorder visited, but note postorder exited at the time that v is postorder exited, construct a path from the current vertex to the root of T.



\section{Ex 3.17}
\begin{figure}[h!]
  \includegraphics[width=\linewidth]{EX17.jpeg}
  \caption{Exe 3.17 Solution.}
  \label{fig:solution 3.17}
\end{figure}

\begin{math}
  \left\{
    \begin{array}{l}
      $T(1) = 1$             \qquad             if \: n \:  = \: 1;\\
      $T(n) = 2T(n/2) - 1$     \qquad      if  \: n  \: is \: even;\\
      $T(n) = 2T((n-1)/2) + 1$    \quad   if  \: n \:  is \: odd;
    \end{array}
  \right.
\end{math}


\section{Ex 3.18}

\subsection{}
The special value should separate the equation , so it has $T(n) = 1024n + 2T(n/2)$ and $T(n/2) = 4(n/2)^2$  and  $T(n) = 4n^2$.
\begin{equation}
1024n + 4(n/2)^2 = 4n^2
\end{equation}
Hence, $n = 512$.


\subsection{}
\begin{figure}[h!]
  \includegraphics[width=\linewidth]{EX18.jpeg}
  \caption{Exe 3.18 Solution.}
  \label{fig:solution 3.18}
\end{figure}
So \quad $ T(n) = 1024n({\log_2 n} -7)$.


\section{Ex 3.19}
\subsection{}
\begin{math}
  \left\{
    \begin{array}{l}
      $L(0) = 2$             \qquad             if \: k \:  = \: 0;\\
      $L(k) = 2 + 2L(k-1)$     \qquad      if  \: k  \: > \: 0;\\
    \end{array}
  \right.
\end{math}

\subsection{}
\begin{math}
  \left\{
    \begin{array}{l}
      $N(0) = 1$             \qquad             if \: k \:  = \: 0;\\
      $N(k) = 3 + 4N(k-1)$     \qquad      if  \: k  \: > \: 0;\\
    \end{array}
  \right.
\end{math}


\subsection{}
 
$L(k) = 2^{k}+2$     \qquad      if  \: k  \: > \: 0;\\
$N(k) = 3+4^{k}$    \qquad      if  \: k  \: > \: 0;\\
$S(k) = N(k)+1+2L(k)$


\section{Ex 3.28}
\subsection{}
$A(n) = n{\log_2 n}+n$


\subsection{}
$B(n) = n{\log_3 n}+n$

\subsection{}
$C(n) = n^2{\log_2 n}+n^2$

\subsection{}
$D(n) = n^2{\log_3 n}+n^2$ 

\subsection{}
\begin{figure}[H]
  \includegraphics[width=\linewidth]{EX28e.jpeg}
  \caption{Exe 3.28e Solution.}
  \label{fig:solution 3.28e}
\end{figure}

\subsection{}
\begin{figure}[H]
  \includegraphics[width=\linewidth]{EX28f.jpeg}
  \caption{Exe 3.28f Solution.}
  \label{fig:solution 3.28f}
\end{figure}

\subsection{}
\subsection{}
\begin{figure}[H]
  \includegraphics[width=\linewidth]{EX28gh.jpeg}
  \caption{Exe 3.28gh Solution.}
  \label{fig:solution 3.28gh}
\end{figure}

\subsection{}
\subsection{}
\begin{figure}[H]
  \includegraphics[width=\linewidth]{EX28ij.jpeg}
  \caption{Exe 3.28ij Solution.}
  \label{fig:solution 3.28ij}
\end{figure}

\subsection{}
$K(n) = n^2({\log_3 n}-4) + 81*9^{{(\log_3 n} - 4}$

\subsection{}
$L(n) = {\log_9 n}\sqrt{n} +\sqrt{n} $

\section{3.general}
\begin{figure}[H]
  \includegraphics[width=\linewidth]{EXgeneral.jpeg}
  \caption{Exe 3.general Solution.}
  \label{fig:solution 3.general}
\end{figure}

























 
 
 
 
 
 
 
 
 
 
 
 
 
 
 



\end{document}
