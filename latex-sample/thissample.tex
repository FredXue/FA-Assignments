%to produce a pdf, you type: pdflatex thissample
\input{00PRELIMS}  %loads 00PRELIMS.tex, which also loads zrogram.tex (all in same directory)

L $T$ be a not-necessarily binary tree where each vertex $v$ in $T$ has the two fields $v.min$ and $v.dat$. 
Suppose that the $.dat$ field of every vertex in $T$ already stores a number.

\noindent
{\bf Problem:} Present an efficient algorithm that for each vertex $v$ in $T$ stores in $v.min$ the minimum of the $.dat$ values in
the subtree rooted by $v$ (including the value $v.dat$).

\noindent
\Prog\qq     % \Prog starts the code environment. The \qq starts a new line without a line number
\Global the tree $T$: its vertex set with preloaded $.dat$ values, and its child lists;\qq
\phantom{\Global }each vertex $v$ in $T$ has the field $v.min$ and preloaded field $v.dat$;\qq
\Function{$MinT$}{;; $v$};\qq     %The ;; is part of the pseudocode syntax, and is explained in Ch 2
\OUTPUT: $v.min$ will equal the minimum of all $.dat$ values in $v$'s subtree;\qq
$v.min\leftarrow v.dat$;\Comment{a preorder initialization}\qq
\Foreach child ${w}$ of $v$ \Do\qq
$v.min\leftarrow \min\{v.min,\,MinT(w)\}$\Comment{use solution for subtree rooted by $w$ to update $v.min$}\qq
\Endfor;\qq
\Return($v.min$)\qq
\Fini %The \Fini ends the code environment and prints the function/procedure name if used. It also adds the ;

The program is initiated with the call $y\leftarrow MinT(T)$.


These problems and solutions illustrate a fundamental property that is typical of many DFS-based solutions to real problems, and more. The
problem description will usually be defined in terms 
of data or records that are in some sense the same for each record.
(For each vertex $v$, store in $v.min$ the smallest value in {\bf any of the many vertices} in the subtree $\cdots$.)
But instead of implementing a ``democratic'' trip to poll each vertex in the tree, the problem was translated into a hierarchical characterization
that used just the local-root vertex $v$ and the subtree-solutions for each of child $w$ of $v$.

\noindent
\Prog\qq
\Global the tree $T$: its vertex set with preloaded $.dat$ values, and its child lists;\qq
\Procedure{$MinT$}{;; $v$};\qq
\OUTPUT: $v.min$ will equal the minimum of all $.dat$ values in $v$'s subtree;\p
%                                           the \p starts a new line with an incremented line number
$v.min\leftarrow v.dat$;\Comment{a preorder initialization}\p
\Foreach child ${w}$ of $v$ \Do\p
$MinT(w)$;\p
$v.min\leftarrow \min\{v.min,\,w.min\}$\Comment{use solution for subtree rooted by $w$ to update $v.min$}\p
\Endfor; \qq
\Fini

This program is initiated with the call $MinT(T)$.

\end{document}
